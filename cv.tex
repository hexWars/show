%---------------------------------------------------------------------------------------------%
%	The MIT License (MIT)
%
%	Copyright (c) 2021 Jitin Nair
%
%	Permission is hereby granted, free of charge, to any person obtaining a copy
%	of this software and associated documentation files (the "Software"), to deal
%	in the Software without restriction, including without limitation the rights
%	to use, copy, modify, merge, publish, distribute, sublicense, and/or sell
%	copies of the Software, and to permit persons to whom the Software is
%	furnished to do so, subject to the following conditions:
%	
%	THE SOFTWARE IS PROVIDED "AS IS", WITHOUT WARRANTY OF ANY KIND, EXPRESS OR
%	IMPLIED, INCLUDING BUT NOT LIMITED TO THE WARRANTIES OF MERCHANTABILITY,
%	FITNESS FOR A PARTICULAR PURPOSE AND NONINFRINGEMENT. IN NO EVENT SHALL THE
%	AUTHORS OR COPYRIGHT HOLDERS BE LIABLE FOR ANY CLAIM, DAMAGES OR OTHER
%	LIABILITY, WHETHER IN AN ACTION OF CONTRACT, TORT OR OTHERWISE, ARISING FROM,
%	OUT OF OR IN CONNECTION WITH THE SOFTWARE OR THE USE OR OTHER DEALINGS IN
%	THE SOFTWARE.
%	
%
%---------------------------------------------------------------------------------------------%



\documentclass[a4paper]{article}


% 数学公式
\usepackage{amsmath}
% 插图
\usepackage{graphicx}
% 颜色
\usepackage{xcolor}
% 表格
\usepackage{array}
% 字体
\usepackage{fontspec}
% 中文
\usepackage[UTF8]{ctex}
% fontawesome图标
\usepackage{fontawesome5}
% 修改\section
\usepackage{titlesec}
% 字体
\usepackage{fontspec}
% 字体大小和行间距
\usepackage{etoolbox}
% 字体格式修改
\usepackage{setspace}
% 画图
% \usepackage{tikz}
% % 修改页面样式
% \usepackage{fancyhdr}
% \pagestyle{fancy}
% \fancyhf{}
% \fancyfoot{}
% \renewcommand{\headrulewidth}{0pt}
% \renewcommand{\footrulewidth}{0pt}
% 页面布局
\usepackage[a4paper]{geometry}
% 超链接颜色修改
\usepackage[unicode, draft=false]{hyperref}

% 自定义
\usepackage{resume}

%-------------------------------------------------------------------------------
%	简历正文
%-------------------------------------------------------------------------------

\begin{document}

% 去掉页码
\pagestyle{empty}
% 去除缩进,article的正文每段会默认缩进
\setlength{\parindent}{0pt}

{\fontsize{9pt}{7pt}\selectfont

%-------------------------------------------------------------------------------
%	社交信息
%-------------------------------------------------------------------------------

\begin{center}
    \textbf{\Huge 蔡伟杰} \\ 
    \vspace{5pt} % 行间距
    \hspace{10pt}
    \href{tel:+8613560851572}{\faPhoneVolume \ 13560851572} | 
    \href{weixin:Cxm71701}{\faWeixin \ Cxm71701} | 
    \href{https://github.com/hexWars}{\faGithub \ hexWars}  | 
    \href{mailto:cxm71701@gmail.com}{\faEnvelope \ cxm71701@gmail.com} | 
    \href{https://blog.sehnsucht.top/}{\faRss \ blog.sehnsucht.top}
    \vspace{-5pt}
\end{center}

%-------------------------------------------------------------------------------
%	教育经历
%-------------------------------------------------------------------------------

\ignorespaces
\section{\textbf{教育经历}}
    \resumeTable
    {北京师范大学珠海分校}{2019年9月 -- 2023年6月}
    {软件工程 \ 本科 \ 信息技术学院}{GPA:3.5(前\%5)}


    % \begin{apart}{6}{0.5}
    %     \textbf{北京师范大学珠海分校} \hfill 2019-2023 \\
    %     软件工程 \ 本科 \ 信息技术学院 \hfill GPA: 3.5(专业前\%5)
    % \end{apart}
    % \ignorespaces
    


%-------------------------------------------------------------------------------
%	专业技能
%-------------------------------------------------------------------------------

\section{\textbf{专业技能}}

    \textbullet 掌握常见数据结构及算法,具有良好的编程习惯 \\
    \textbullet 熟悉SpringBoot和Node.js进行后台开发 \\
    \textbullet 熟悉Linux常见命令以及Git等工具的使用 \\
    \textbullet 熟悉HTML,CSS,JavaScript的使用 \\
    \textbullet 熟悉JSP,Thymeleaf,FreeMarker等模板引擎 \\
    \textbullet 了解Jenkins持续集成,自动化部署,使用过Jmeter和Selenium \\
    \textbullet 了解Docker容器原理并拥有使用经验 \\
    \textbullet 了解Redis以及其简单使用 \\

    大一开始在CSDN发表技术文章,至今约300篇,浏览量约13w,全站排名最高约\textbf{5000}, \href{https://blog.csdn.net/Dueser}{博客地址}

%-------------------------------------------------------------------------------
%	荣誉奖项
%-------------------------------------------------------------------------------

\section{\textbf{荣誉奖项}}

    \textbullet 大学生创新创业项目(国家级)两项 \\
    \textbullet 专业一等奖学金(三次),校优秀学生干部(一次) \\
    \textbullet 北京师范大学珠海校区“互联网++”大学生创新创业大赛“红旅创意组”金奖 \hfill 2022年6月 \\
    \textbullet 国家软件著作权登记证书(一项)\hfill 2022年5月 \\
    \textbullet 十三届蓝桥杯省赛一等奖 \hfill 2022年4月 \\
    \textbullet PAT乙级满分(1/982) \hfill 2021年12月 \\
    \textbullet 第九届“逐鹿今朝”大学生创业计划竞赛(校赛)三等奖 \hfill 2021年12月 \\
    \textbullet (第三届)算法设计与编程挑战赛(秋季赛)团队赛组铜奖 \hfill 2021年11月 \\
    \textbullet (第三届)算法设计与编程挑战赛(秋季赛)个人赛组银奖 \hfill 2021年11月

\ignorespaces

%-------------------------------------------------------------------------------
%	社团和组织经历
%-------------------------------------------------------------------------------

\section{\textbf{社团和组织经历}}
    
    \resumeTable
    {程序设计爱好者协会}{2020年10月 -- 2021年10月}
    {会长}{}
    负责校内程序设计竞赛的出题和测题以及协会招新,协助学院举办“远光杯”粤澳计算机程序设计大赛等多项大型活动

    \vspace{2pt} 同时管理宣传部,技术部,竞赛部,组织部近40人,任期结束时在学生社团评比中所在社团获得总评\textbf{第一}的成绩 


%-------------------------------------------------------------------------------
%	实习经验
%-------------------------------------------------------------------------------

\section{\textbf{实习经验}}
    \resumeTable
    {中科迪宏广州研发中心}{2023年2月 -- 至今}
    {后端工程师}{广州}
    
    xxxxxxxxxxxxxxxxxxx

%-------------------------------------------------------------------------------
%	项目经历
%-------------------------------------------------------------------------------

\section{\textbf{项目经历}}
    \resumeTable
    {Node.js个人博客系统}{2021年10月 -- 2021年11月}
    {全栈开发}{}
    \begin{minipage}{17cm}\vspace{2pt}
        \textbf{技术选型:}Node.js+Express+EJS+Mysql+Docker
    \end{minipage}
    

    \begin{minipage}{17cm}\vspace{2pt}
        \textbf{项目描述:}本项目针对个人用户发布博客的场景进行开发工作,其中包括发布删除修改查找博客,指定社交信息展示,分类目录,登录模块权限管理,登录日志等功能。从数据库表设计开始,文章管理和目录分类接口的实现 ,最后打包成Docker镜像方便以Docker方式部署
    \end{minipage}

    \begin{minipage}{17cm}\vspace{2pt}
        \textbf{职责描述:}负责项目的总体设计,参考Halo博客进行原型设计,完成路由模块代码和前端代码,并制作Docker镜像上传
    \end{minipage}



%-------------------------------------------------------------------------------
%	自我评价
%-------------------------------------------------------------------------------


\section{\textbf{自我评价}}
执着,喜欢思考,爱好广泛,拥有冷静的内心和炽热的灵魂,独特的思维角度 \\
从第一次AC开始,从第一次发博客开始,在思考中不断学习,希望能和更加优秀的人一起共事 \\
热爱开源,热爱技术,对代码的美感深深的吸引

\vfill
\center{\footnotesize Last updated: \today}

\end{document}
